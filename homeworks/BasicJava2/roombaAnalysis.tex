\documentclass[paper=a4, fontsize=11pt, parskip=full]{scrartcl} % A4 paper and 11pt font size

\usepackage[T1]{fontenc} % Use 8-bit encoding that has 256 glyphs
\usepackage{fourier} % Use the Adobe Utopia font for the document - comment this line to return to the LaTeX default
\usepackage[english]{babel} % English language/hyphenation
\usepackage{amsmath,amsfonts,amsthm} % Math packages

\usepackage{graphicx}

\usepackage{float}

%Preamble
\usepackage{listings}
\usepackage{color}
\definecolor{javared}{rgb}{0.6,0,0} % for strings
\definecolor{javagreen}{rgb}{0.25,0.5,0.35} % comments
\definecolor{javapurple}{rgb}{0.5,0,0.35} % keywords
\definecolor{javadocblue}{rgb}{0.25,0.35,0.75} % javadoc

\lstset{language=Java,
basicstyle=\ttfamily,
keywordstyle=\color{javapurple}\bfseries,
stringstyle=\color{javared},
commentstyle=\color{javagreen},
morecomment=[s][\color{javadocblue}]{/**}{*/},
numbers=left,
numberstyle=\tiny\color{black},
stepnumber=2,
numbersep=10pt,
tabsize=4,
showspaces=false,
showstringspaces=false}


\usepackage{hyperref}
\hypersetup{
    colorlinks=true,
    linkcolor=blue,
    filecolor=magenta,
    urlcolor=cyan,
}

\usepackage{lipsum} % Used for inserting dummy 'Lorem ipsum' text into the template

\usepackage{sectsty} % Allows customizing section commands
\allsectionsfont{\centering \normalfont\scshape} % Make all sections centered, the default font and small caps

\usepackage{fancyhdr} % Custom headers and footers
\pagestyle{fancyplain} % Makes all pages in the document conform to the custom headers and footers
\fancyhead{} % No page header - if you want one, create it in the same way as the footers below
\fancyfoot[L]{} % Empty left footer
\fancyfoot[C]{} % Empty center footer
\fancyfoot[R]{\thepage} % Page numbering for right footer
\renewcommand{\headrulewidth}{0pt} % Remove header underlines
\renewcommand{\footrulewidth}{0pt} % Remove footer underlines
\setlength{\headheight}{13.6pt} % Customize the height of the header

\numberwithin{equation}{section} % Number equations within sections (i.e. 1.1, 1.2, 2.1, 2.2 instead of 1, 2, 3, 4)
\numberwithin{figure}{section} % Number figures within sections (i.e. 1.1, 1.2, 2.1, 2.2 instead of 1, 2, 3, 4)
\numberwithin{table}{section} % Number tables within sections (i.e. 1.1, 1.2, 2.1, 2.2 instead of 1, 2, 3, 4)

\setlength\parindent{0pt} % Removes all indentation from paragraphs - comment this line for an assignment with lots of text

%----------------------------------------------------------------------------------------
%	TITLE SECTION
%----------------------------------------------------------------------------------------

\newcommand{\horrule}[1]{\rule{\linewidth}{#1}} % Create horizontal rule command with 1 argument of height

\title{
\normalfont \normalsize
\textsc{University of Virginia, Department of Computer Science} \\ [25pt] % Your university, school and/or department name(s)
\horrule{0.5pt} \\[0.4cm] % Thin top horizontal rule
\huge Basic Java 2 - Roomba Analysis \\ % The assignment title
\horrule{2pt} \\[0.5cm] % Thick bottom horizontal rule
}

\author{Nada Basit and Mark Floryan}

\date{\normalsize\today} % Today's date or a custom date

\begin{document}

\maketitle % Print the title

%------------------------------------------------


%----------------------------------------------------------------------------------------
%	Analysis
%----------------------------------------------------------------------------------------
\section{Summary}

The goal of this homework is to write a report comparing two distinct strategies for your Roomba simulator. You will be writing many reports this semester, so this is a good opportunity to get used to the kind of results / reports we expect moving forward this semester.

\begin{enumerate}
	\item Construct at least two distinct strategies for your makeMove() method.
	\item Run an experiment comparing how well each strategy cleans various rooms.
	\item Write a report summarizing and analyzing your findings.
	\item \textbf{FILES TO DOWNLOAD:} None
	\item \textbf{FILES TO SUBMIT:} RoombaAnalysis.pdf
\end{enumerate}

\subsection{Performing an Experiment}

Your first task is to write \textbf{two unique} makeMove() methods that incorporate different strategies for cleaning the rooms. You should be able to argue that the two strategies are significantly different AND that it is not obvious to you which is better. You should be geniuinely curious which approach will succeed. It is okay if your second makeMove() method is a change / variation of your first, as long as they are substantially different in some way.

You should then plan to test your two Roombas on several rooms, given the following constraints:

\begin{itemize}
	\item You will use four total room configurations. two of the rooms will be smaller in size and two will be larger in size (you may choose the exact sizes).
	\item For each set of two above, one will have a small number of obstacles generated, and one will have a large number of obstacles generated.
	\item You should run each Roomba in each of the four configurations at least 5 times each. That is $5 * 4 = 20$ total executions minimum.
	\item For each room configuration, you should average the percentage of the room cleaned for each Roomba strategy.
\end{itemize}

\subsection{Report}

Summarize your experiment and your findings in a report. Make sure to adhere to these general guidelines:

\begin{itemize}
	\item Your submission MUST BE a pdf document. You will receive a zero if it is not.
	\item Your document MUST be presented as if submitted to a professional publication outlet. You can use the \href{https://uva-cs.github.io/dsa1/homeworks/WordPaperTemplate.zip}{template} posted in the course repository or follow \href{https://www.springer.com/us/computer-science/lncs/conference-proceedings-guidelines}{Springer's guidelines for conference proceedings}.
	\item You should write your report as if it is original novel research. \emph{Yes, this is not really original research per se, but we'd like you to get in the habit of writing as if you are communicating new findings to others.}
	\item The grammar / spelling / professionalism of this document should be sound.
	\item When possible, do not use the first person. Instead of "I ran the code 60 times", use "The code was executed 60 times...".
\end{itemize}

In addition to the general guidelines above, please follow the following rough outline for your paper:

\begin{itemize}
	\item \textbf{Abstract}: Summarize the entire document in a single paragraph/
	\item \textbf{Introduction}: Present the problem, and provide details regarding the two strategies you implemented.
	\item \textbf{Methods}: Describe your methodology for collecting data. How many rooms, how many executions, how you averaged things, etc.
	\item \textbf{Results}: Describe your results from your execution runs.
	\item \textbf{Conclusion}: Interpret your results. Which strategy was better? Why was it better? Were you surprised? Was one strategy better in some situations and not in others? Why do you think that is? Notice that I'm not looking for a particular answer here. Show me that you can interpret what happened when you ran your code.
\end{itemize}

Lastly, your paper MUST contain the following things:

\begin{itemize}
	\item A figure showing the relevant code for each of your makeMove() methods.
	\item A table (methods section) summarizing the different experimental groups and how many execution runs were done in each group.
	\item A table (results section) summarizing each experimental group and the average percent cleaned for each (as well as any other data you decided to collect).
	\item Some kind of graph visualizing the results of the table from the previous bullet.
\end{itemize}

%------------------------------------------------

%----------------------------------------------------------------------------------------

\end{document}


%----------------------------------------------------------------------------------------
%----------------------------------------------------------------------------------------
%----------------------------------------------------------------------------------------
%----------------------------------------------------------------------------------------
%----------------------------------------------------------------------------------------
%----------------------------------------------------------------------------------------


%WORKS CITED:

%%%%%%%%%%%%%%%%%%%%%%%%%%%%%%%%%%%%%%%%%
% Short Sectioned Assignment
% LaTeX Template
% Version 1.0 (5/5/12)
%
% This template has been downloaded from:
% http://www.LaTeXTemplates.com
%
% Original author:
% Frits Wenneker (http://www.howtotex.com)
%
% License:
% CC BY-NC-SA 3.0 (http://creativecommons.org/licenses/by-nc-sa/3.0/)
%
%%%%%%%%%%%%%%%%%%%%%%%%%%%%%%%%%%%%%%%%%

%----------------------------------------------------------------------------------------
%	PACKAGES AND OTHER DOCUMENT CONFIGURATIONS
%----------------------------------------------------------------------------------------
