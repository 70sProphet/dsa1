\documentclass[paper=a4, fontsize=11pt, parskip=full]{scrartcl} % A4 paper and 11pt font size

\usepackage[T1]{fontenc} % Use 8-bit encoding that has 256 glyphs
\usepackage{fourier} % Use the Adobe Utopia font for the document - comment this line to return to the LaTeX default
\usepackage[english]{babel} % English language/hyphenation
\usepackage{amsmath,amsfonts,amsthm} % Math packages

\usepackage{graphicx}

\usepackage{float}

%Preamble
\usepackage{listings}
\usepackage{color}
\definecolor{javared}{rgb}{0.6,0,0} % for strings
\definecolor{javagreen}{rgb}{0.25,0.5,0.35} % comments
\definecolor{javapurple}{rgb}{0.5,0,0.35} % keywords
\definecolor{javadocblue}{rgb}{0.25,0.35,0.75} % javadoc
 
\lstset{language=Java,
basicstyle=\ttfamily,
keywordstyle=\color{javapurple}\bfseries,
stringstyle=\color{javared},
commentstyle=\color{javagreen},
morecomment=[s][\color{javadocblue}]{/**}{*/},
numbers=left,
numberstyle=\tiny\color{black},
stepnumber=2,
numbersep=10pt,
tabsize=4,
showspaces=false,
showstringspaces=false}


\usepackage{hyperref}
\hypersetup{
    colorlinks=true,
    linkcolor=blue,
    filecolor=magenta,      
    urlcolor=cyan,
}

\usepackage{lipsum} % Used for inserting dummy 'Lorem ipsum' text into the template

\usepackage{sectsty} % Allows customizing section commands
\allsectionsfont{\centering \normalfont\scshape} % Make all sections centered, the default font and small caps

\usepackage{fancyhdr} % Custom headers and footers
\pagestyle{fancyplain} % Makes all pages in the document conform to the custom headers and footers
\fancyhead{} % No page header - if you want one, create it in the same way as the footers below
\fancyfoot[L]{} % Empty left footer
\fancyfoot[C]{} % Empty center footer
\fancyfoot[R]{\thepage} % Page numbering for right footer
\renewcommand{\headrulewidth}{0pt} % Remove header underlines
\renewcommand{\footrulewidth}{0pt} % Remove footer underlines
\setlength{\headheight}{13.6pt} % Customize the height of the header

\numberwithin{equation}{section} % Number equations within sections (i.e. 1.1, 1.2, 2.1, 2.2 instead of 1, 2, 3, 4)
\numberwithin{figure}{section} % Number figures within sections (i.e. 1.1, 1.2, 2.1, 2.2 instead of 1, 2, 3, 4)
\numberwithin{table}{section} % Number tables within sections (i.e. 1.1, 1.2, 2.1, 2.2 instead of 1, 2, 3, 4)

\setlength\parindent{0pt} % Removes all indentation from paragraphs - comment this line for an assignment with lots of text

%----------------------------------------------------------------------------------------
%	TITLE SECTION
%----------------------------------------------------------------------------------------

\newcommand{\horrule}[1]{\rule{\linewidth}{#1}} % Create horizontal rule command with 1 argument of height

\title{	
\normalfont \normalsize 
\textsc{University of Virginia, Department of Computer Science} \\ [25pt] % Your university, school and/or department name(s)
\horrule{0.5pt} \\[0.4cm] % Thin top horizontal rule
\huge Basic Java 1 - Four Functions \\ % The assignment title
\horrule{2pt} \\[0.5cm] % Thick bottom horizontal rule
}

\author{Nada Basit and Mark Floryan}

\date{\normalsize\today} % Today's date or a custom date

\begin{document}

\maketitle % Print the title

%----------------------------------------------------------------------------------------
%	Four Functions
%----------------------------------------------------------------------------------------
\section{Summary}

The goal of this homework is to continue practicing writing some simple java functions. You will do the following:

\begin{enumerate}
	\item Implement the four methods shown below.
	\item Implement a small program to allow a user input values to pass to each of the four methods below.
	\item \textbf{FILES TO DOWNLOAD:} None
	\item \textbf{FILE TO SUBMIT:} FourFunctions.java
\end{enumerate}

\subsection{Four Functions}

For this homework, you will implement six functions, as well as a small program that allows a user to select which method they'd like to test. You will start by implementing the following methods:

\begin{enumerate}
\item \textbf{average}: Accepts an array of integers as a parameter, and returns the average of the numbers as a double.
\item \textbf{median}: Accepts an array of integers as a parameter, and returns the \href{https://en.wikipedia.org/wiki/Median}{median} of the numbers. \emph{Note that you can invoke your max and min function here to make this easier. You don't necessarily need to sort the data.} 
\item \textbf{stddev}: Accepts an array of integers as a parameter, and return the \href{https://en.wikipedia.org/wiki/Standard_deviation}{standard deviation} of the numbers. \emph{Note that you can invoke your average method here.}
\item \textbf{mode}: Accepts and array of integers as a parameter, and returns the most commonly occurring value among them. \emph{If there is more than one mode, you may return any of them.}
\end{enumerate}

Once these methods are complete, you should write a program that asks the user to input a number one through four. If the user doesn't enter a valid number, the program exits. If the user enters a valid number one through four, then they are prompted to enter five integers. Those seven integers are then given to the corresponding method above (1 = average, 2 = median, 3 = stddev, 4 = mode) and the answer (a single value) is printed to the console. The program then exits.

\textbf{IMPORTANT NOTE: The average, median, and standard deviation should be printed as a double to two decimal places. The mode should be printed as an integer. You can handle the floating point values using this line of code:}

\begin{lstlisting}
System.out.printf("%.2f\n", average(nums));
\end{lstlisting}

You should submit one file for this homework, \textbf{FourFunctions.java}.

\textbf{SAMPLE INPUT:}

\begin{lstlisting}
1
3
5
9
7
11
\end{lstlisting}

\textbf{SAMPLE OUTPUT:}

\begin{lstlisting}
7.00
\end{lstlisting}

You should submit one file for this homework, \textbf{FourFunctions.java}.

\subsection{Gradescope}

You should submit your code to \emph{Gradescope}. If you are having trouble with your submission, you should double check the following common problems:

\begin{enumerate}
	\item Make sure you are only submitting one file, and it is called \emph{FourFunctions.java} exactly.
	\item Make sure you remove any \emph{package} statements from your code before submitting. The autograder doesn't expect your file to be inside a package for this assignment.
	\item Make sure your output is in the correct format (see above) exactly. You should not be printing ANYTHING else or the autograder will think your output is incorrect.
\end{enumerate}

%------------------------------------------------

%----------------------------------------------------------------------------------------

\end{document}


%----------------------------------------------------------------------------------------
%----------------------------------------------------------------------------------------
%----------------------------------------------------------------------------------------
%----------------------------------------------------------------------------------------
%----------------------------------------------------------------------------------------
%----------------------------------------------------------------------------------------


%WORKS CITED:

%%%%%%%%%%%%%%%%%%%%%%%%%%%%%%%%%%%%%%%%%
% Short Sectioned Assignment
% LaTeX Template
% Version 1.0 (5/5/12)
%
% This template has been downloaded from:
% http://www.LaTeXTemplates.com
%
% Original author:
% Frits Wenneker (http://www.howtotex.com)
%
% License:
% CC BY-NC-SA 3.0 (http://creativecommons.org/licenses/by-nc-sa/3.0/)
%
%%%%%%%%%%%%%%%%%%%%%%%%%%%%%%%%%%%%%%%%%

%----------------------------------------------------------------------------------------
%	PACKAGES AND OTHER DOCUMENT CONFIGURATIONS
%----------------------------------------------------------------------------------------