\documentclass[paper=a4, fontsize=11pt, parskip=full]{scrartcl} % A4 paper and 11pt font size

\usepackage[T1]{fontenc} % Use 8-bit encoding that has 256 glyphs
\usepackage{fourier} % Use the Adobe Utopia font for the document - comment this line to return to the LaTeX default
\usepackage[english]{babel} % English language/hyphenation
\usepackage{amsmath,amsfonts,amsthm} % Math packages

\usepackage{graphicx}

\usepackage{float}

%Preamble
\usepackage{listings}
\usepackage{color}
\definecolor{javared}{rgb}{0.6,0,0} % for strings
\definecolor{javagreen}{rgb}{0.25,0.5,0.35} % comments
\definecolor{javapurple}{rgb}{0.5,0,0.35} % keywords
\definecolor{javadocblue}{rgb}{0.25,0.35,0.75} % javadoc

\lstset{language=Java,
basicstyle=\ttfamily,
keywordstyle=\color{javapurple}\bfseries,
stringstyle=\color{javared},
commentstyle=\color{javagreen},
morecomment=[s][\color{javadocblue}]{/**}{*/},
numbers=left,
numberstyle=\tiny\color{black},
stepnumber=2,
numbersep=10pt,
tabsize=4,
showspaces=false,
showstringspaces=false}


\usepackage{hyperref}
\hypersetup{
    colorlinks=true,
    linkcolor=blue,
    filecolor=magenta,
    urlcolor=cyan,
}

\usepackage{lipsum} % Used for inserting dummy 'Lorem ipsum' text into the template

\usepackage{sectsty} % Allows customizing section commands
\allsectionsfont{\centering \normalfont\scshape} % Make all sections centered, the default font and small caps

\usepackage{fancyhdr} % Custom headers and footers
\pagestyle{fancyplain} % Makes all pages in the document conform to the custom headers and footers
\fancyhead{} % No page header - if you want one, create it in the same way as the footers below
\fancyfoot[L]{} % Empty left footer
\fancyfoot[C]{} % Empty center footer
\fancyfoot[R]{\thepage} % Page numbering for right footer
\renewcommand{\headrulewidth}{0pt} % Remove header underlines
\renewcommand{\footrulewidth}{0pt} % Remove footer underlines
\setlength{\headheight}{13.6pt} % Customize the height of the header

\numberwithin{equation}{section} % Number equations within sections (i.e. 1.1, 1.2, 2.1, 2.2 instead of 1, 2, 3, 4)
\numberwithin{figure}{section} % Number figures within sections (i.e. 1.1, 1.2, 2.1, 2.2 instead of 1, 2, 3, 4)
\numberwithin{table}{section} % Number tables within sections (i.e. 1.1, 1.2, 2.1, 2.2 instead of 1, 2, 3, 4)

\setlength\parindent{0pt} % Removes all indentation from paragraphs - comment this line for an assignment with lots of text

%----------------------------------------------------------------------------------------
%	TITLE SECTION
%----------------------------------------------------------------------------------------

\newcommand{\horrule}[1]{\rule{\linewidth}{#1}} % Create horizontal rule command with 1 argument of height

\title{
\normalfont \normalsize
\textsc{University of Virginia, Department of Computer Science} \\ [25pt] % Your university, school and/or department name(s)
\horrule{0.5pt} \\[0.4cm] % Thin top horizontal rule
\huge BSTs - Tree Implementation \\ % The assignment title
\horrule{2pt} \\[0.5cm] % Thick bottom horizontal rule
}

\author{Nada Basit and Mark Floryan}

\date{\normalsize\today} % Today's date or a custom date

\begin{document}

\maketitle % Print the title

%----------------------------------------------------------------------------------------
%	Summary
%----------------------------------------------------------------------------------------

\section{Summary}

For this homework, you will be implementing two classes to create a working Binary Search Tree. You will start by implementing a basic Binary Tree with the three tree traversals covered in class. Then, you will implement and test a working Binary Search Tree.

\begin{enumerate}
	\item Download the provided starter code. You can \textbf{ignore the AVLTree} class. You will be using that one next week.
	\item Implement some general methods in the BinaryTree class.
	\item Implement the BinarySearchTree class
	\item Use the provided tester files to verify your implementation works. Note that you should test your code more so than the provided tester does this time. The tester is NOT as thorough as in previous homeworks.
	\item \textbf{FILES TO DOWNLOAD:} \href{https://uva-cs.github.io/dsa1/homeworks/BinarySearchTrees/code/BinarySearchTrees.zip}{BinarySearchTrees.zip}
	\item \textbf{FILES TO SUBMIT:} BinaryTree.java, BinarySearchTree.java
\end{enumerate}

%------------------------------------------------

\subsection{BinaryTree.java}

To begin, implement the BinaryTree class. These are methods that are useful for ANY binary tree (whether balanced or not). You will need to implement the following methods (\textbf{NOTE: These methods DO NO print anything, they return the traversals as a string.}):

\begin{lstlisting}
public class BinaryTree<T>{
	//Returns a string representing the tree nodes
	//given as an in-order traversal
	//All on one line, space separated
	private String printInOrder(TreeNode<T> curNode);
	
	//Returns a string representing the tree nodes
	//given as an in-order traversal
	//All on one line, space separated
	private String printPreOrder(TreeNode<T> curNode);
	
	//Returns a string representing the tree nodes
	//given as an in-order traversal
	//All on one line, space separated
	private String printPostOrder(TreeNode<T> curNode);
}
\end{lstlisting}

The Binary Tree contains a few methods that are implemented for you. This includes the main print functions, that simply call the helper functions described above on the root to give off the recursive prints. In addition, a method called printTree() is provided that prints the tree in a somewhat formatted method. This may be useful when debugging your code. Lastly, a simple recursive method that computes the height of the binary tree is provided as an example of recursion in trees.

\subsection{BinarySearchTree.java}

Next, you will implement a binary search tree. This class will extend the binary tree class from earlier, and thus inherit the print methods defined earlier. Your binary search tree should implement the provided Tree interface, shown below. 

\begin{lstlisting}
public interface Tree<T extends Comparable<T>> {
	
	//remember to IGNORE DUPLICATES
	public void insert(T data);
	
	public boolean find(T data);

	public void remove(T data);
}

/* You BST implements the interface above */
public class BinarySearchTree<T extends Comparable<T>>
					extends BinaryTree<T> implements Tree<T>{

		//TODO: Implement this class
}
\end{lstlisting}

You may add other supporting methods to your binary search tree if you find that to be helpful. 

\subsection{Testing your code}

Once you are done, you can look at the provided tester file. This tester is VERY minimal and only checks one single test case, providing you with the expected output. You can (and should) create more tests and manually check them to ensure your tree is operating correctly.

You should submit \textbf{two files} for this homework: \textbf{BinaryTree.java, and BinarySearchTree.java}.

\subsection{Gradescope}

You should submit your code to \emph{Gradescope}. If you are having trouble with your submission, you should double check the following common problems:

\begin{enumerate}
	\item Make sure you are only submitting the two requested files, and they are named \emph{BinaryTree.java} and \emph{BinarySearchTree.java} exactly.
	\item Make sure you keep any \emph{package} statements in your code before submitting. The autograder expects your files to have the package statements that are provided in the downloaded project.
	\item Make sure your output is in the correct format. You should not be printing ANYTHING else or the autograder will think your output is incorrect.
\end{enumerate}


%------------------------------------------------


%----------------------------------------------------------------------------------------

\end{document}


%----------------------------------------------------------------------------------------
%----------------------------------------------------------------------------------------
%----------------------------------------------------------------------------------------
%----------------------------------------------------------------------------------------
%----------------------------------------------------------------------------------------
%----------------------------------------------------------------------------------------


%WORKS CITED:

%%%%%%%%%%%%%%%%%%%%%%%%%%%%%%%%%%%%%%%%%
% Short Sectioned Assignment
% LaTeX Template
% Version 1.0 (5/5/12)
%
% This template has been downloaded from:
% http://www.LaTeXTemplates.com
%
% Original author:
% Frits Wenneker (http://www.howtotex.com)
%
% License:
% CC BY-NC-SA 3.0 (http://creativecommons.org/licenses/by-nc-sa/3.0/)
%
%%%%%%%%%%%%%%%%%%%%%%%%%%%%%%%%%%%%%%%%%

%----------------------------------------------------------------------------------------
%	PACKAGES AND OTHER DOCUMENT CONFIGURATIONS
%----------------------------------------------------------------------------------------

