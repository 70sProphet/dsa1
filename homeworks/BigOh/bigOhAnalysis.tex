\documentclass[paper=a4, fontsize=11pt, parskip=full]{scrartcl} % A4 paper and 11pt font size

\usepackage[T1]{fontenc} % Use 8-bit encoding that has 256 glyphs
\usepackage{fourier} % Use the Adobe Utopia font for the document - comment this line to return to the LaTeX default
\usepackage[english]{babel} % English language/hyphenation
\usepackage{amsmath,amsfonts,amsthm} % Math packages

\usepackage{graphicx}

\usepackage{float}

%Preamble
\usepackage{listings}
\usepackage{color}
\definecolor{javared}{rgb}{0.6,0,0} % for strings
\definecolor{javagreen}{rgb}{0.25,0.5,0.35} % comments
\definecolor{javapurple}{rgb}{0.5,0,0.35} % keywords
\definecolor{javadocblue}{rgb}{0.25,0.35,0.75} % javadoc

\lstset{language=Java,
basicstyle=\ttfamily,
keywordstyle=\color{javapurple}\bfseries,
stringstyle=\color{javared},
commentstyle=\color{javagreen},
morecomment=[s][\color{javadocblue}]{/**}{*/},
numbers=left,
numberstyle=\tiny\color{black},
stepnumber=2,
numbersep=10pt,
tabsize=4,
showspaces=false,
showstringspaces=false}


\usepackage{hyperref}
\hypersetup{
    colorlinks=true,
    linkcolor=blue,
    filecolor=magenta,
    urlcolor=cyan,
}

\usepackage{lipsum} % Used for inserting dummy 'Lorem ipsum' text into the template

\usepackage{sectsty} % Allows customizing section commands
\allsectionsfont{\centering \normalfont\scshape} % Make all sections centered, the default font and small caps

\usepackage{fancyhdr} % Custom headers and footers
\pagestyle{fancyplain} % Makes all pages in the document conform to the custom headers and footers
\fancyhead{} % No page header - if you want one, create it in the same way as the footers below
\fancyfoot[L]{} % Empty left footer
\fancyfoot[C]{} % Empty center footer
\fancyfoot[R]{\thepage} % Page numbering for right footer
\renewcommand{\headrulewidth}{0pt} % Remove header underlines
\renewcommand{\footrulewidth}{0pt} % Remove footer underlines
\setlength{\headheight}{13.6pt} % Customize the height of the header

\numberwithin{equation}{section} % Number equations within sections (i.e. 1.1, 1.2, 2.1, 2.2 instead of 1, 2, 3, 4)
\numberwithin{figure}{section} % Number figures within sections (i.e. 1.1, 1.2, 2.1, 2.2 instead of 1, 2, 3, 4)
\numberwithin{table}{section} % Number tables within sections (i.e. 1.1, 1.2, 2.1, 2.2 instead of 1, 2, 3, 4)

\setlength\parindent{0pt} % Removes all indentation from paragraphs - comment this line for an assignment with lots of text

%----------------------------------------------------------------------------------------
%	TITLE SECTION
%----------------------------------------------------------------------------------------

\newcommand{\horrule}[1]{\rule{\linewidth}{#1}} % Create horizontal rule command with 1 argument of height

\title{
\normalfont \normalsize
\textsc{University of Virginia, Department of Computer Science} \\ [25pt] % Your university, school and/or department name(s)
\horrule{0.5pt} \\[0.4cm] % Thin top horizontal rule
\huge Big Oh - Analyzing Scalability of Algorithms \\ % The assignment title
\horrule{2pt} \\[0.5cm] % Thick bottom horizontal rule
}

\author{Nada Basit and Mark Floryan}

\date{\normalsize\today} % Today's date or a custom date

\begin{document}

\maketitle % Print the title

%----------------------------------------------------------------------------------------
%	Scalability of Algorithms
%----------------------------------------------------------------------------------------

\section{Summary}

For this homework, you will use the methods you wrote in the pre-lab to understand the scalability of algorithms, and the differences between common runtimes.

\begin{enumerate}
	\item Reacquaint yourself with the five methods (and one provided method) from the Big-OH implementation homework.
	\item Run each method on increasing array sizes. Use completion times to fill out the BigOh chart.
	\item \textbf{FILES TO DOWNLOAD:} None (\emph{Use your code from the previous assignment})
	\item \textbf{FILES TO SUBMIT:} BigOh.pdf
\end{enumerate}

%------------------------------------------------

\subsection{Grab your code from previous assignment}

In the previous assignment, you implemented five methods (one one was provided to you). The headers for these methods are provided again here for your convenience.

\begin{lstlisting}
// Searches for item in sorted array a.
// Returns true iff item is found in a
// Should run in Theta(logn) time
public static binarySearch(int[] a, int item);

// Finds the largest item in a (a might not be sorted)
// Should run in Theta(n) time
public static int max(int[] a);

// Invokes binarySearch() with a.length random numbers
// Returns how many times those random numbers were found in a
// Should run in Theta(nlogn) time
public static int multipleBinarySearch(int[] a);

// Counts how many pairs of items in a sum to a multiple of 5
// Should run in Theta(n^2) time
public static int allPairs(int[] a);

// Counts how many a,b,c combinations there are in a such that a+b=c
// Should run in Theta(n^3) time
public static int allTriads(int[] a);

// loops through all the subsets of array a
// e..g, {1,2,3} would print {},{1},{2},{3},{1,2},{1,3},{2,3},{1,2,3}
// Should run in Theta(2^n) time
public static int allSubsets(int[] a);
\end{lstlisting}

%------------------------------------------------

\subsection{Analyzing Scalability}

Your task is to analyze how large inputs can reasonable get at each runtime. We have provided a main function to you that asks for two inputs: The method you would like to execute, and how large you'd like the input size (array) to be. The method also provides some code that times how long the operation takes and reports the time to you.

Run the code multiple times and fill out the chart below. Submit the chart in a pdf file (BigOh.pdf). As you are filling out this chart, think about how much larger the input can get as your code becomes more and more efficient. If a time takes more than 1 minute, don't wait for the code to finish, just report that the code took more than 60000 ms. Please report all times in ms (a time of 0 ms is fine if the code finishes quickly). 

\renewcommand{\arraystretch}{2.0}
\begin{tabular}{|r|l|l|l|l|l|l|l|l|l|}
\hline
Input Size:                        & 1 & 10 & 100 & 1,000 & 10,000 & 10\textasciicircum{}5 & 10\textasciicircum{}6 & 10\textasciicircum{}7 & 10\textasciicircum{}8 \\ \hline
Binary Search (logn)               &   &    &     &       &        &                       &                       &                       &                       \\ \hline
Max (n)                            &   &    &     &       &        &                       &                       &                       &                       \\ \hline
Multi Binary Search (nlogn)        &   &    &     &       &        &                       &                       &                       &                       \\ \hline
All Pairs (n\textasciicircum{}2)   &   &    &     &       &        &                       &                       &                       &                       \\ \hline
All Triads (n\textasciicircum{}3)  &   &    &     &       &        &                       &                       &                       &                       \\ \hline
All Subsets (2\textasciicircum{}n) &   &    &     &       &        &                       &                       &                       &                       \\ \hline
\end{tabular}


You should submit one file for this homework: \textbf{BigOh.pdf}.

\subsection{Analysis}

In your report, write a couple of paragraphs about the results you see. Do they match what you expected? How large can inputs get before certain methods start to slow down to an unreasonable pace? Does anything about the times you recorded stick out as strange? 


\subsection{What if I didn't finish the previous homework?}

For this homework, we do not care if your methods are returning the \textbf{correct answer}, but we DO care that they have the correct asymptotic complexity. So, you may have to do some work to at least ensure the latter. Make sure each method is doing the correct amount of work, even if it has a bug or small error in it. Then, do the analysis without worrying about the correctness of the output.


\subsection{Gradescope}

You should submit your pdf to \emph{Gradescope}. There is no autograder for this assignment, and a grader will be manually checking your submission.


%----------------------------------------------------------------------------------------

\end{document}


%----------------------------------------------------------------------------------------
%----------------------------------------------------------------------------------------
%----------------------------------------------------------------------------------------
%----------------------------------------------------------------------------------------
%----------------------------------------------------------------------------------------
%----------------------------------------------------------------------------------------


%WORKS CITED:

%%%%%%%%%%%%%%%%%%%%%%%%%%%%%%%%%%%%%%%%%
% Short Sectioned Assignment
% LaTeX Template
% Version 1.0 (5/5/12)
%
% This template has been downloaded from:
% http://www.LaTeXTemplates.com
%
% Original author:
% Frits Wenneker (http://www.howtotex.com)
%
% License:
% CC BY-NC-SA 3.0 (http://creativecommons.org/licenses/by-nc-sa/3.0/)
%
%%%%%%%%%%%%%%%%%%%%%%%%%%%%%%%%%%%%%%%%%

%----------------------------------------------------------------------------------------
%	PACKAGES AND OTHER DOCUMENT CONFIGURATIONS
%----------------------------------------------------------------------------------------
