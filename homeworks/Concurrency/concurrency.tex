\documentclass[paper=a4, fontsize=11pt, parskip=full]{scrartcl} % A4 paper and 11pt font size

\usepackage[T1]{fontenc} % Use 8-bit encoding that has 256 glyphs
\usepackage{fourier} % Use the Adobe Utopia font for the document - comment this line to return to the LaTeX default
\usepackage[english]{babel} % English language/hyphenation
\usepackage{amsmath,amsfonts,amsthm} % Math packages

\usepackage{graphicx}

\usepackage{float}

%Preamble
\usepackage{listings}
\usepackage{color}
\definecolor{javared}{rgb}{0.6,0,0} % for strings
\definecolor{javagreen}{rgb}{0.25,0.5,0.35} % comments
\definecolor{javapurple}{rgb}{0.5,0,0.35} % keywords
\definecolor{javadocblue}{rgb}{0.25,0.35,0.75} % javadoc

\lstset{language=Java,
basicstyle=\ttfamily,
keywordstyle=\color{javapurple}\bfseries,
stringstyle=\color{javared},
commentstyle=\color{javagreen},
morecomment=[s][\color{javadocblue}]{/**}{*/},
numbers=left,
numberstyle=\tiny\color{black},
stepnumber=2,
numbersep=10pt,
tabsize=4,
showspaces=false,
showstringspaces=false}


\usepackage{hyperref}
\hypersetup{
    colorlinks=true,
    linkcolor=blue,
    filecolor=magenta,
    urlcolor=cyan,
}

\usepackage{lipsum} % Used for inserting dummy 'Lorem ipsum' text into the template

\usepackage{sectsty} % Allows customizing section commands
\allsectionsfont{\centering \normalfont\scshape} % Make all sections centered, the default font and small caps

\usepackage{fancyhdr} % Custom headers and footers
\pagestyle{fancyplain} % Makes all pages in the document conform to the custom headers and footers
\fancyhead{} % No page header - if you want one, create it in the same way as the footers below
\fancyfoot[L]{} % Empty left footer
\fancyfoot[C]{} % Empty center footer
\fancyfoot[R]{\thepage} % Page numbering for right footer
\renewcommand{\headrulewidth}{0pt} % Remove header underlines
\renewcommand{\footrulewidth}{0pt} % Remove footer underlines
\setlength{\headheight}{13.6pt} % Customize the height of the header

\numberwithin{equation}{section} % Number equations within sections (i.e. 1.1, 1.2, 2.1, 2.2 instead of 1, 2, 3, 4)
\numberwithin{figure}{section} % Number figures within sections (i.e. 1.1, 1.2, 2.1, 2.2 instead of 1, 2, 3, 4)
\numberwithin{table}{section} % Number tables within sections (i.e. 1.1, 1.2, 2.1, 2.2 instead of 1, 2, 3, 4)

\setlength\parindent{0pt} % Removes all indentation from paragraphs - comment this line for an assignment with lots of text

%----------------------------------------------------------------------------------------
%	TITLE SECTION
%----------------------------------------------------------------------------------------

\newcommand{\horrule}[1]{\rule{\linewidth}{#1}} % Create horizontal rule command with 1 argument of height

\title{
\normalfont \normalsize
\textsc{University of Virginia, Department of Computer Science} \\ [25pt] % Your university, school and/or department name(s)
\horrule{0.5pt} \\[0.4cm] % Thin top horizontal rule
\huge Concurrency - Concurrent Queue Implementation \\ % The assignment title
\horrule{2pt} \\[0.5cm] % Thick bottom horizontal rule
}

\author{Nada Basit and Mark Floryan}

\date{\normalsize\today} % Today's date or a custom date

\begin{document}

\maketitle % Print the title

%----------------------------------------------------------------------------------------
%	Summary
%----------------------------------------------------------------------------------------

\section{Summary}

For this homework, you will be implementing a simple concurrent queue. For this assignment, we are going to focus on correctness (i.e., the queue must work with multiple threads without crashing). Optimizing the speed of your concurrent queue is optional.

\begin{enumerate}
	\item Download the provided starter code
	\item Implement the ConcurrentQueue class
	\item Run the main in MainTester.java to ensure your queue works correctly
	\item \textbf{FILES TO DOWNLOAD:} \href{https://uva-cs.github.io/dsa1/homeworks/Concurrency/code/concurrency.zip}{concurrency.zip}
	\item \textbf{FILES TO SUBMIT:} concurrentQueue.zip
\end{enumerate}

%------------------------------------------------

\subsection{ConcurrentQueue.java}

To begin, implement the *ConcurrentQueue* class inside the ConcurrentQueue.java file. The methods you are responsible for are listed below. This Queue \textbf{must be a linked-list based queue}. You may use Java's built-in Linked List (import java.util.LinkedList) or you may use your own implementation from the previous homeworks.

\begin{lstlisting}
	public class Queue<T>{
		public Queue();

		public int size();

		public void enqueue(T data);

		public T dequeue();
	}
\end{lstlisting}

Your class must be alterable by different threads. Thus, you should protect the fields (head, tail, etc.) of your linked list from being corrupted by multiple threads accessing the queue at once. For this homework, we care about *correctness* (i.e., the queue works as intended with multiple threads even if it could be optimized to run faster).

\subsection{MainTester.java}

You can test your code by running the main method in *MainTester.java*. This tester will first test your queue using a single thread and time the results. Then, the method will test your queue again using two threads and time the results again. Any errors that occur should be printed to the console. Make sure you run the tester multiple times. Race conditions can sometimes cause code to appear to work but not consistently.

\subsection{OPTIONAL: Optimizing}

Although we only care about correctness for this homework, if you are interested you should try to make your queue run as quickly as possible. What can you do to increase the degree to which multiple threads can use the queue in parallel? How fast can you make the concurrent test that we provided?

When you are done, submit your entire project as a zip file to Collab.

%------------------------------------------------




%----------------------------------------------------------------------------------------

\end{document}


%----------------------------------------------------------------------------------------
%----------------------------------------------------------------------------------------
%----------------------------------------------------------------------------------------
%----------------------------------------------------------------------------------------
%----------------------------------------------------------------------------------------
%----------------------------------------------------------------------------------------


%WORKS CITED:

%%%%%%%%%%%%%%%%%%%%%%%%%%%%%%%%%%%%%%%%%
% Short Sectioned Assignment
% LaTeX Template
% Version 1.0 (5/5/12)
%
% This template has been downloaded from:
% http://www.LaTeXTemplates.com
%
% Original author:
% Frits Wenneker (http://www.howtotex.com)
%
% License:
% CC BY-NC-SA 3.0 (http://creativecommons.org/licenses/by-nc-sa/3.0/)
%
%%%%%%%%%%%%%%%%%%%%%%%%%%%%%%%%%%%%%%%%%

%----------------------------------------------------------------------------------------
%	PACKAGES AND OTHER DOCUMENT CONFIGURATIONS
%----------------------------------------------------------------------------------------
